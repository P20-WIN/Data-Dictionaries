% Options for packages loaded elsewhere
\PassOptionsToPackage{unicode}{hyperref}
\PassOptionsToPackage{hyphens}{url}
%
\documentclass[
]{article}
\usepackage{amsmath,amssymb}
\usepackage{lmodern}
\usepackage{iftex}
\ifPDFTeX
  \usepackage[T1]{fontenc}
  \usepackage[utf8]{inputenc}
  \usepackage{textcomp} % provide euro and other symbols
\else % if luatex or xetex
  \usepackage{unicode-math}
  \defaultfontfeatures{Scale=MatchLowercase}
  \defaultfontfeatures[\rmfamily]{Ligatures=TeX,Scale=1}
\fi
% Use upquote if available, for straight quotes in verbatim environments
\IfFileExists{upquote.sty}{\usepackage{upquote}}{}
\IfFileExists{microtype.sty}{% use microtype if available
  \usepackage[]{microtype}
  \UseMicrotypeSet[protrusion]{basicmath} % disable protrusion for tt fonts
}{}
\makeatletter
\@ifundefined{KOMAClassName}{% if non-KOMA class
  \IfFileExists{parskip.sty}{%
    \usepackage{parskip}
  }{% else
    \setlength{\parindent}{0pt}
    \setlength{\parskip}{6pt plus 2pt minus 1pt}}
}{% if KOMA class
  \KOMAoptions{parskip=half}}
\makeatother
\usepackage{xcolor}
\usepackage[margin=1in]{geometry}
\usepackage{color}
\usepackage{fancyvrb}
\newcommand{\VerbBar}{|}
\newcommand{\VERB}{\Verb[commandchars=\\\{\}]}
\DefineVerbatimEnvironment{Highlighting}{Verbatim}{commandchars=\\\{\}}
% Add ',fontsize=\small' for more characters per line
\usepackage{framed}
\definecolor{shadecolor}{RGB}{248,248,248}
\newenvironment{Shaded}{\begin{snugshade}}{\end{snugshade}}
\newcommand{\AlertTok}[1]{\textcolor[rgb]{0.94,0.16,0.16}{#1}}
\newcommand{\AnnotationTok}[1]{\textcolor[rgb]{0.56,0.35,0.01}{\textbf{\textit{#1}}}}
\newcommand{\AttributeTok}[1]{\textcolor[rgb]{0.77,0.63,0.00}{#1}}
\newcommand{\BaseNTok}[1]{\textcolor[rgb]{0.00,0.00,0.81}{#1}}
\newcommand{\BuiltInTok}[1]{#1}
\newcommand{\CharTok}[1]{\textcolor[rgb]{0.31,0.60,0.02}{#1}}
\newcommand{\CommentTok}[1]{\textcolor[rgb]{0.56,0.35,0.01}{\textit{#1}}}
\newcommand{\CommentVarTok}[1]{\textcolor[rgb]{0.56,0.35,0.01}{\textbf{\textit{#1}}}}
\newcommand{\ConstantTok}[1]{\textcolor[rgb]{0.00,0.00,0.00}{#1}}
\newcommand{\ControlFlowTok}[1]{\textcolor[rgb]{0.13,0.29,0.53}{\textbf{#1}}}
\newcommand{\DataTypeTok}[1]{\textcolor[rgb]{0.13,0.29,0.53}{#1}}
\newcommand{\DecValTok}[1]{\textcolor[rgb]{0.00,0.00,0.81}{#1}}
\newcommand{\DocumentationTok}[1]{\textcolor[rgb]{0.56,0.35,0.01}{\textbf{\textit{#1}}}}
\newcommand{\ErrorTok}[1]{\textcolor[rgb]{0.64,0.00,0.00}{\textbf{#1}}}
\newcommand{\ExtensionTok}[1]{#1}
\newcommand{\FloatTok}[1]{\textcolor[rgb]{0.00,0.00,0.81}{#1}}
\newcommand{\FunctionTok}[1]{\textcolor[rgb]{0.00,0.00,0.00}{#1}}
\newcommand{\ImportTok}[1]{#1}
\newcommand{\InformationTok}[1]{\textcolor[rgb]{0.56,0.35,0.01}{\textbf{\textit{#1}}}}
\newcommand{\KeywordTok}[1]{\textcolor[rgb]{0.13,0.29,0.53}{\textbf{#1}}}
\newcommand{\NormalTok}[1]{#1}
\newcommand{\OperatorTok}[1]{\textcolor[rgb]{0.81,0.36,0.00}{\textbf{#1}}}
\newcommand{\OtherTok}[1]{\textcolor[rgb]{0.56,0.35,0.01}{#1}}
\newcommand{\PreprocessorTok}[1]{\textcolor[rgb]{0.56,0.35,0.01}{\textit{#1}}}
\newcommand{\RegionMarkerTok}[1]{#1}
\newcommand{\SpecialCharTok}[1]{\textcolor[rgb]{0.00,0.00,0.00}{#1}}
\newcommand{\SpecialStringTok}[1]{\textcolor[rgb]{0.31,0.60,0.02}{#1}}
\newcommand{\StringTok}[1]{\textcolor[rgb]{0.31,0.60,0.02}{#1}}
\newcommand{\VariableTok}[1]{\textcolor[rgb]{0.00,0.00,0.00}{#1}}
\newcommand{\VerbatimStringTok}[1]{\textcolor[rgb]{0.31,0.60,0.02}{#1}}
\newcommand{\WarningTok}[1]{\textcolor[rgb]{0.56,0.35,0.01}{\textbf{\textit{#1}}}}
\usepackage{graphicx}
\makeatletter
\def\maxwidth{\ifdim\Gin@nat@width>\linewidth\linewidth\else\Gin@nat@width\fi}
\def\maxheight{\ifdim\Gin@nat@height>\textheight\textheight\else\Gin@nat@height\fi}
\makeatother
% Scale images if necessary, so that they will not overflow the page
% margins by default, and it is still possible to overwrite the defaults
% using explicit options in \includegraphics[width, height, ...]{}
\setkeys{Gin}{width=\maxwidth,height=\maxheight,keepaspectratio}
% Set default figure placement to htbp
\makeatletter
\def\fps@figure{htbp}
\makeatother
\setlength{\emergencystretch}{3em} % prevent overfull lines
\providecommand{\tightlist}{%
  \setlength{\itemsep}{0pt}\setlength{\parskip}{0pt}}
\setcounter{secnumdepth}{-\maxdimen} % remove section numbering
\ifLuaTeX
  \usepackage{selnolig}  % disable illegal ligatures
\fi
\IfFileExists{bookmark.sty}{\usepackage{bookmark}}{\usepackage{hyperref}}
\IfFileExists{xurl.sty}{\usepackage{xurl}}{} % add URL line breaks if available
\urlstyle{same} % disable monospaced font for URLs
\hypersetup{
  pdftitle={server.R},
  pdfauthor={WonderlyC},
  hidelinks,
  pdfcreator={LaTeX via pandoc}}

\title{server.R}
\author{WonderlyC}
\date{2023-02-28}

\begin{document}
\maketitle

\begin{Shaded}
\begin{Highlighting}[]
\DocumentationTok{\#\# P20 WIN DATA DICTIONARY SERVER SCRIPT \#\#}
\CommentTok{\# The Server Script contains the instructions for how to build your filters and table (i.e. how they interact and what data to display)}
\CommentTok{\# Shiny Tutorials and other resources: https://shiny.rstudio.com/tutorial/}

\DocumentationTok{\#\#\# Loading Required Packages (install may be required) \#\#\#\#}
\FunctionTok{library}\NormalTok{(shiny)}
\FunctionTok{library}\NormalTok{(shinyWidgets)}
\FunctionTok{library}\NormalTok{(readxl)}
\FunctionTok{library}\NormalTok{(tidyverse)}
\end{Highlighting}
\end{Shaded}

\begin{verbatim}
## -- Attaching packages --------------------------------------- tidyverse 1.3.2 --
## v ggplot2 3.4.0      v purrr   1.0.1 
## v tibble  3.1.8      v dplyr   1.0.10
## v tidyr   1.2.1      v stringr 1.5.0 
## v readr   2.1.3      v forcats 0.5.2 
## -- Conflicts ------------------------------------------ tidyverse_conflicts() --
## x dplyr::filter() masks stats::filter()
## x dplyr::lag()    masks stats::lag()
\end{verbatim}

\begin{Shaded}
\begin{Highlighting}[]
\FunctionTok{library}\NormalTok{(shinydashboard)}
\end{Highlighting}
\end{Shaded}

\begin{verbatim}
## 
## Attaching package: 'shinydashboard'
## 
## The following object is masked from 'package:graphics':
## 
##     box
\end{verbatim}

\begin{Shaded}
\begin{Highlighting}[]
\FunctionTok{library}\NormalTok{(shinythemes)}
\FunctionTok{library}\NormalTok{(bslib)}
\end{Highlighting}
\end{Shaded}

\begin{verbatim}
## 
## Attaching package: 'bslib'
## 
## The following object is masked from 'package:utils':
## 
##     page
\end{verbatim}

\begin{Shaded}
\begin{Highlighting}[]
\FunctionTok{library}\NormalTok{(DT)}
\end{Highlighting}
\end{Shaded}

\begin{verbatim}
## 
## Attaching package: 'DT'
## 
## The following objects are masked from 'package:shiny':
## 
##     dataTableOutput, renderDataTable
\end{verbatim}

\begin{Shaded}
\begin{Highlighting}[]
\FunctionTok{library}\NormalTok{(htmlwidgets)}
\NormalTok{thematic}\SpecialCharTok{::}\FunctionTok{thematic\_shiny}\NormalTok{(}\AttributeTok{font =} \StringTok{"auto"}\NormalTok{)}

\DocumentationTok{\#\#\#\# Reading in the Data Dictionary \#\#\#\#}
\NormalTok{P20WIN\_Data\_Dictionary }\OtherTok{\textless{}{-}} \FunctionTok{read\_excel}\NormalTok{(}\StringTok{"P20WIN\_Data\_Dictionary.xlsx"}\NormalTok{)}

\DocumentationTok{\#\#\#\# Server Script \#\#\#\#}
\NormalTok{server }\OtherTok{\textless{}{-}} \ControlFlowTok{function}\NormalTok{(session, input, output) \{}
  
  \CommentTok{\# The following chunk is helpful when you are testing your app before publishing. If you don\textquotesingle{}t include this your app may not stop running if there are }
  \CommentTok{\# issues in your script. This function ensures that if you close the app the script will stop running.}
  
  \CommentTok{\#session$onSessionEnded(function() \{}
  \CommentTok{\#  stopApp()}
  \CommentTok{\# \})}
  
  \CommentTok{\# The following section is taking the input filters created in the ui file and making them reactive to each other. }
  \CommentTok{\# These filters do need to have a set hierarchy, meaning they should go from most generic to most specific, in this case agency {-}\textgreater{} program {-}\textgreater{} data category.}
  
  \CommentTok{\# The first filter event is if the agency filter is used, when this event is observed, the available choices for both the program and data category filters}
  \CommentTok{\# are updated to only display the choices that correspond to the agency/agencies selected. }

\NormalTok{  agencies }\OtherTok{\textless{}{-}} \FunctionTok{reactive}\NormalTok{(\{}
    \FunctionTok{filter}\NormalTok{(P20WIN\_Data\_Dictionary, Agency }\SpecialCharTok{\%in\%}\NormalTok{ input}\SpecialCharTok{$}\NormalTok{select\_agency)}
\NormalTok{  \})}
  \FunctionTok{observeEvent}\NormalTok{(}\FunctionTok{agencies}\NormalTok{(), \{}
\NormalTok{    choices\_1 }\OtherTok{\textless{}{-}} \FunctionTok{unique}\NormalTok{(}\FunctionTok{agencies}\NormalTok{()}\SpecialCharTok{$}\NormalTok{Program)}
\NormalTok{    choices\_2 }\OtherTok{\textless{}{-}} \FunctionTok{unique}\NormalTok{(}\FunctionTok{agencies}\NormalTok{()}\SpecialCharTok{$}\StringTok{\textasciigrave{}}\AttributeTok{Data Category}\StringTok{\textasciigrave{}}\NormalTok{)}
    \FunctionTok{updatePickerInput}\NormalTok{(session,}
                      \AttributeTok{inputId =} \StringTok{"select\_program"}\NormalTok{,}
                      \AttributeTok{choices=}\NormalTok{choices\_1,}
                      \AttributeTok{selected =}\NormalTok{ choices\_1)}
    \FunctionTok{updatePickerInput}\NormalTok{(session,}
                      \AttributeTok{inputId =} \StringTok{"select\_category"}\NormalTok{,}
                      \AttributeTok{choices =}\NormalTok{ choices\_2,}
                      \AttributeTok{selected =}\NormalTok{ choices\_2)}
\NormalTok{  \})}
  
  \CommentTok{\# The second filter event is if the program filter is used, when this event is observed, the available choices for the data category filter}
  \CommentTok{\# are updated to only display the choices that correspond to the program/programs selected. }
  
\NormalTok{  programs }\OtherTok{\textless{}{-}} \FunctionTok{reactive}\NormalTok{(\{}
    \FunctionTok{filter}\NormalTok{(}\FunctionTok{agencies}\NormalTok{(), Program }\SpecialCharTok{\%in\%}\NormalTok{ input}\SpecialCharTok{$}\NormalTok{select\_program)}
\NormalTok{  \})}
  \FunctionTok{observeEvent}\NormalTok{(}\FunctionTok{programs}\NormalTok{(), \{}
\NormalTok{    choices\_3 }\OtherTok{\textless{}{-}} \FunctionTok{unique}\NormalTok{(}\FunctionTok{programs}\NormalTok{()}\SpecialCharTok{$}\StringTok{\textasciigrave{}}\AttributeTok{Data Category}\StringTok{\textasciigrave{}}\NormalTok{)}
    \FunctionTok{updatePickerInput}\NormalTok{(session,}
                      \AttributeTok{inputId =} \StringTok{"select\_category"}\NormalTok{,}
                      \AttributeTok{choices =}\NormalTok{ choices\_3,}
                      \AttributeTok{selected =}\NormalTok{ choices\_3)}
\NormalTok{  \})}
  
  \CommentTok{\# It should be noted that filters will not work in the other direction. If you create have a "select agency" filter that updates the program filter then you cannot have}
  \CommentTok{\# a "select program" filter that updates the agency filter because you will end up with a never ending loop where both filters are simultaneously trying to update the other. }
  
  
  \CommentTok{\# The rest of the script is setting up the data table/data dictionary.}
  \CommentTok{\# The first six lines are naming the input filters as data filters for the table.}

\NormalTok{  output}\SpecialCharTok{$}\NormalTok{mytable }\OtherTok{\textless{}{-}}\NormalTok{ DT}\SpecialCharTok{::}\FunctionTok{renderDataTable}\NormalTok{(\{}
    \FunctionTok{req}\NormalTok{(input}\SpecialCharTok{$}\NormalTok{select\_program)}
    \FunctionTok{req}\NormalTok{(input}\SpecialCharTok{$}\NormalTok{select\_category)}
    \FunctionTok{agencies}\NormalTok{() }\SpecialCharTok{\%\textgreater{}\%}
      \FunctionTok{filter}\NormalTok{(}\StringTok{\textasciigrave{}}\AttributeTok{Data Category}\StringTok{\textasciigrave{}} \SpecialCharTok{\%in\%}\NormalTok{ input}\SpecialCharTok{$}\NormalTok{select\_category) }\SpecialCharTok{\%\textgreater{}\%}
      \FunctionTok{filter}\NormalTok{(Program }\SpecialCharTok{\%in\%}\NormalTok{ input}\SpecialCharTok{$}\NormalTok{select\_program)}
\NormalTok{    \},}
    \AttributeTok{escape =} \DecValTok{0}\NormalTok{,}
    \AttributeTok{options =} \FunctionTok{list}\NormalTok{(}
                \AttributeTok{scrollX =} \ConstantTok{FALSE}\NormalTok{, }\DocumentationTok{\#\# disable scrolling on X axis}
                \AttributeTok{scrollY =} \StringTok{\textquotesingle{}64vh\textquotesingle{}}\NormalTok{,   }\DocumentationTok{\#\# enable scrolling on Y axis and setting table height to 64\% of the screens verticle height. This will depend on how tall the navbar or header is.}
                \AttributeTok{autoWidth =} \ConstantTok{FALSE}\NormalTok{, }\DocumentationTok{\#\# disable auto column widths (set below)}
                \AttributeTok{columnDefs =} \FunctionTok{list}\NormalTok{(}
                  \FunctionTok{list}\NormalTok{(}\AttributeTok{targets =} \FunctionTok{c}\NormalTok{(}\DecValTok{0}\NormalTok{), }\AttributeTok{visible =} \ConstantTok{TRUE}\NormalTok{, }\AttributeTok{width =} \StringTok{\textquotesingle{}60px\textquotesingle{}}\NormalTok{, }\DocumentationTok{\#\# this column will work as a button to expand the row to show hidden fields (see callback = JS below)}
                       \AttributeTok{orderable =} \ConstantTok{FALSE}\NormalTok{, }\AttributeTok{className =} \StringTok{\textquotesingle{}details{-}control\textquotesingle{}}\NormalTok{), }
                  \FunctionTok{list}\NormalTok{(}\AttributeTok{targets =} \FunctionTok{c}\NormalTok{(}\DecValTok{1}\NormalTok{,}\DecValTok{2}\NormalTok{,}\DecValTok{3}\NormalTok{), }\AttributeTok{visible =} \ConstantTok{TRUE}\NormalTok{, }\AttributeTok{width=}\StringTok{\textquotesingle{}89px\textquotesingle{}}\NormalTok{), }
                  \FunctionTok{list}\NormalTok{(}\AttributeTok{targets =} \FunctionTok{c}\NormalTok{(}\DecValTok{4}\NormalTok{), }\AttributeTok{visible =} \ConstantTok{TRUE}\NormalTok{, }\AttributeTok{width=}\StringTok{\textquotesingle{}215px\textquotesingle{}}\NormalTok{), }
                  \FunctionTok{list}\NormalTok{(}\AttributeTok{targets =} \FunctionTok{c}\NormalTok{(}\DecValTok{5}\NormalTok{), }\AttributeTok{visible =} \ConstantTok{TRUE}\NormalTok{), }
                  \FunctionTok{list}\NormalTok{(}\AttributeTok{targets =} \FunctionTok{c}\NormalTok{(}\DecValTok{6}\NormalTok{,}\DecValTok{7}\NormalTok{), }\AttributeTok{visible =} \ConstantTok{FALSE}\NormalTok{, }\AttributeTok{width =} \StringTok{\textquotesingle{}0px\textquotesingle{}}\NormalTok{)), }\DocumentationTok{\#\# hiding the columns that will be shown when the expansion button is clicked}
                \AttributeTok{paging =} \ConstantTok{FALSE}\NormalTok{,    }\DocumentationTok{\#\# disable paginate so that the table shows as a continuous scroll}
                \AttributeTok{pageLength =} \DecValTok{200000}\NormalTok{, }\DocumentationTok{\#\# number of rows to output for each page. We chose a large number that should not ever be reached to ensure no elements are dropped.}
                \AttributeTok{server =} \ConstantTok{FALSE}\NormalTok{,   }\DocumentationTok{\#\# use client{-}side processing}
                \AttributeTok{dom =} \StringTok{\textquotesingle{}Bfrt\textquotesingle{}}\NormalTok{,}
                \AttributeTok{buttons =} \FunctionTok{list}\NormalTok{( }\DocumentationTok{\#\# adding download buttons for CSV and Excel documents. Only the selected elements will be downloaded. }
                  \FunctionTok{list}\NormalTok{(}\AttributeTok{extend =} \StringTok{\textquotesingle{}csv\textquotesingle{}}\NormalTok{, }\AttributeTok{title =} \ConstantTok{NULL}\NormalTok{, }\AttributeTok{exportOptions =} \FunctionTok{list}\NormalTok{(}\AttributeTok{columns =} \FunctionTok{c}\NormalTok{(}\DecValTok{1}\SpecialCharTok{:}\DecValTok{7}\NormalTok{)),}
                       \AttributeTok{filename =} \StringTok{"P20\_WIN\_Data\_Dictionary"}\NormalTok{), }
                  \FunctionTok{list}\NormalTok{(}\AttributeTok{extend =} \StringTok{\textquotesingle{}excel\textquotesingle{}}\NormalTok{, }\AttributeTok{title =} \ConstantTok{NULL}\NormalTok{, }\AttributeTok{exportOptions =} \FunctionTok{list}\NormalTok{(}\AttributeTok{columns =} \FunctionTok{c}\NormalTok{(}\DecValTok{1}\SpecialCharTok{:}\DecValTok{7}\NormalTok{)),}
                        \AttributeTok{filename =} \StringTok{"P20\_WIN\_Data\_Dictionary"}\NormalTok{)}
\NormalTok{                 )}
\NormalTok{                ),}
    \DocumentationTok{\#\# custom javascript to create the expansion button and show the hidden child rows. }
    \AttributeTok{callback =} \FunctionTok{JS}\NormalTok{(}\StringTok{" }
\StringTok{  table.column(0).nodes().to$().css(\{cursor: \textquotesingle{}pointer\textquotesingle{}\});}
\StringTok{  var format = function(d) \{}
\StringTok{    return \textquotesingle{}\textless{}div style=}\SpecialCharTok{\textbackslash{}"}\StringTok{background{-}color:\#Ffffff; padding: .5em;}\SpecialCharTok{\textbackslash{}"}\StringTok{\textgreater{} Data Type: \textquotesingle{} +}
\StringTok{            d[6] + \textquotesingle{}\textless{}br\textgreater{} Years Available: \textquotesingle{} + d[7] + \textquotesingle{}\textless{}/div\textgreater{}\textquotesingle{};}
\StringTok{  \};}
\StringTok{  table.on(\textquotesingle{}click\textquotesingle{}, \textquotesingle{}td.details{-}control\textquotesingle{}, function() \{}
\StringTok{    var td = $(this), row = table.row(td.closest(\textquotesingle{}tr\textquotesingle{}));}
\StringTok{    if (row.child.isShown()) \{}
\StringTok{      row.child.hide();}
\StringTok{      td.html(\textquotesingle{}\&oplus;\textquotesingle{});}
\StringTok{    \} else \{}
\StringTok{      row.child(format(row.data())).show();}
\StringTok{      td.html(\textquotesingle{}\&CircleMinus;\textquotesingle{});}
\StringTok{    \}}
\StringTok{  \});"}
\NormalTok{),}
    \AttributeTok{extensions =} \StringTok{\textquotesingle{}Buttons\textquotesingle{}}\NormalTok{,}
    \AttributeTok{selection =} \StringTok{\textquotesingle{}single\textquotesingle{}}\NormalTok{, }\DocumentationTok{\#\# enable selection of a single row}
    \AttributeTok{rownames =} \ConstantTok{FALSE}      \DocumentationTok{\#\# don\textquotesingle{}t show row numbers/names}
\NormalTok{  )}
\NormalTok{\}}
\end{Highlighting}
\end{Shaded}


\end{document}
